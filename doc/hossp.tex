\documentclass[fleqn,10pt]{article}
\usepackage{amsmath}
\usepackage{amssymb}

% type user-defined commands here

\title{Hard random instances generator for the open shop scheduling problem}   % type title between braces
%\author{Tom Scavo}         % type author(s) between braces

\begin{document}
\flushbottom
\maketitle
\thispagestyle{empty}

The random instances generator takes the number of jobs $n$ and the number of machines $m$ as input and requires three parameters $k$, $p$ and $f$.

An instance is generated in two steps, the first step creates the $ n \times m $ matrix $P$ of processing times using the parameter $k$ such that all elements of $P$ are equal to $\dfrac{k}{m}$ (rounded down to the nearest integer). The second step performs a number of perturbations $p$ to the matrix $P$.

A parameter $f$ is used to calculate $v$: a fixed value \textit{to be subtracted} and $mv$: a \textit{maximum subtractable} value. The parameters $k$, $p$ and $f$ are described by:

\begin{itemize}
    \item $k$: integer number, such that the sum of each line of $P$ is equal to $k$. The value $k \mod m$ is added to the diagonal of $P$. 
    
    \item $p$: number of perturbations, such that for each perturbation, two task's processing times $P_{ij}$ and $ P_{kl}$, $i \neq k $ and $j \neq l$, are randomly selected from $P$, from which is subtracted the value of $v$.
    
    \item $f=\{f \in \mathbb{R} \mid 0<f<1\}$, such that:
    \subitem $mv =  \max\{P_{ij},P_{kl}\}-1$ (subtract 1 to avoid tasks of length 0).      
    \subitem $v = (f \cdot mv) + r $, with $r$: a random value between $0$ and $f \cdot mv$. 
\end{itemize}

The value subtracted from $P_{ij}$ and $P_{kl}$ is added to $P_{il}$ and $P_{kj}$ to keep the sum of all rows equal to $k$.

For numerous generated instances the parameters $k$, $p$ and $f$ are randomly generated as follows:

\begin{itemize}
    \item $k$: random value between $ n \times m$ and $n \times m \times 100 $
    \item $p$: random value between $ n \times m$ and $ n \times m^{2} $
    \item $f$: random value between 0 and 1
\end{itemize}


Note that in the case of $n \neq m$, $k \mod m$ is added the diagonal of top-down square sub-matrices of size $\min\{n,m\}$ of $P$. The main diagonal is used on the remaining rectangle ensuring the sum of all rows of $P$ are equal to $k$. 

\begin{thebibliography}{1}
    \bibitem{gp1999} Guéret, C., \& Prins, C. (1999). A new lower bound for the open-shop problem. Annals of Operations Research, 92, 165–183. https://doi.org/10.1023/A:1018930613891.

    %
\end{thebibliography}

\end{document}