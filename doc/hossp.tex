\documentclass[fleqn,10pt]{article}
\usepackage{amsmath}

% type user-defined commands here

\title{Hard random instances generator for the open shop scheduling problem}   % type title between braces
%\author{Tom Scavo}         % type author(s) between braces

\begin{document}
\flushbottom
\maketitle
\thispagestyle{empty}

The random instances generator takes the number of jobs $n$ and the number of machine $m$ as input and require three parameters $k$, $p$ and $f$.
An instance is generated in two steps, the first step creates the $ m \times n $ matrix $P$ of processing time with a parameters $k$. The second step performs a number of perturbations $p$ to the $ m \times n $ matrix $P$ in combination with a parameter $f$.

Parameters $k$, $p$ and $f$ are described by:

\begin{itemize}
    \item $k$: integer number such that the sum of each line of the processing times matrix is equal to $k$. The value $k \mod m$ is added to the diagonal of $P$. 
    \item $p$: number of perturbations. Such that, for each perturbation, two task's processing times $P_{ij}$ and $ P_{kl}$, $i \neq k $ and $j \neq l$, are randomly selected in $P$ for which is subtracted a fixed value calculated with the parameter $f$.
    \item $f$: a ratio used to calculate the fixed value \textit{to be subtracted} from The maximum \textit{substractable} processing time. The maximum removable processing time corresponds to the minimum of the two randomly selected tasks $P_{ij}$ and $ P_{kl}$ (minus 1 to avoid creating tasks of length 0). A random value (between zero and the difference between \textit{substractable} and \textit{to be subtracted}) values is also added to that fixed value.
\end{itemize}

The value removed from $p_{ij}$ and $p_{kl}$ is added to the $p_{il}$ and $p_{kj}$ to keep the sum of all rows equal to $k$.
For numerous generated instances the parameters $k$, $p$ and $f$ are randomly generated as follows:

\begin{itemize}
    \item $k$: random value between $ n \times m$ and $n \times m \times 100 $
    \item $p$: random value between $ n \times m$ and $ n \times m^{2} $
    \item $f$: random value between 0 and 1
\end{itemize}


\begin{thebibliography}{1}
    \bibitem{gp1999} Guéret, C., \& Prins, C. (1999). A new lower bound for the open-shop problem. Annals of Operations Research, 92, 165–183. https://doi.org/10.1023/A:1018930613891.

    %
\end{thebibliography}

\end{document}