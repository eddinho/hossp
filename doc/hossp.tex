\documentclass[11pt]{article}

\textwidth 165mm \textheight 230mm \topmargin -5mm
\oddsidemargin 0mm

% nicer text finish
%\usepackage{kantlipsum}
\usepackage{url}
\usepackage{tgpagella}
\usepackage{microtype}

% multiple figures
\usepackage[tableposition=top]{caption}
\usepackage{subcaption}

%\usepackage{graphicx,xcolor}

\usepackage{xcolor}
\usepackage[color=gray!50]{background}

%Excel2Table
\usepackage{multirow}

%keep figures in place with H
\usepackage{float}

% for highlighting changes
\usepackage{changes}

% clickable citations
%\usepackage{hyperref}
% activate to hide links
%\usepackage[hidelinks]{hyperref}
\usepackage{authblk}            
\renewcommand\Affilfont{\fontsize{8.7}{10.8}\itshape}
\newcommand\CorresAuth{\footnotemark[\arabic{footnote}]}
\renewcommand*{\Authands}{, }
\newcommand{\email}[1]{\texttt{\small #1}}

%for bottomrule and tablenote
\usepackage{booktabs}
\usepackage{amstext}
\usepackage{amssymb}
\usepackage{amsmath}
\usepackage{babel} 

% silence font warning 
\usepackage{textcomp}

\usepackage{algorithm,algpseudocode}
\algnewcommand{\algorithmicforeach}{\textbf{for each}}
\algdef{SE}[FOR]{ForAll}{EndForAll}[1]
  {\algorithmicforall\ #1\ \algorithmicdo}% \ForEach{#1}
  {\algorithmicend\ \algorithmicforall}% \EndForEach
  \algdef{SE}[FOR]{ForEach}{EndForEach}[1]
  {\algorithmicforeach\ #1\ \algorithmicdo}% \ForEach{#1}
  {\algorithmicend\ \algorithmicforeach}% \EndForEach

%% Use the option review to obtain double line spacing
%% \documentclass[preprint,review,12pt]{elsarticle}

%% Use the options 1p,twocolumn; 3p; 3p,twocolumn; 5p; or 5p,twocolumn
%% for a journal layout:
%% \documentclass[final,1p,times]{elsarticle}
%% \documentclass[final,1p,times,twocolumn]{elsarticle}
%% \documentclass[final,3p,times]{elsarticle}
%% \documentclass[final,3p,times,twocolumn]{elsarticle}
%% \documentclass[final,5p,times]{elsarticle}
%% \documentclass[final,5p,times,twocolumn]{elsarticle}

%% if you use PostScript figures in your article
%% use the graphics package for simple commands
%% \usepackage{graphics}
%% or use the graphicx package for more complicated commands
%% \usepackage{graphicx}
%% or use the epsfig package if you prefer to use the old commands
%% \usepackage{epsfig}

\DeclareUnicodeCharacter{2212}{-}

%% The amssymb package provides various useful mathematical symbols
\usepackage{amssymb}
%% The amsthm package provides extended theorem environments
\usepackage{mathtools}% http://ctan.org/pkg/mathtools
\usepackage{amsmath,systeme} % equations
\usepackage{amsthm}
\newtheorem{theorem}{Theorem}[section]
\newtheorem{corollary}{Corollary}[theorem]
\newtheorem{lemma}[theorem]{Lemma}
\newtheorem{definition}{Definition}[section]

\usepackage[normalem]{ulem}               % to striketrhourhg text
\newcommand\redout{\bgroup\markoverwith
{\textcolor{red}{\rule[0.5ex]{2pt}{0.8pt}}}\ULon}

%% The lineno packages adds line numbers. Start line numbering with
%% \begin{linenumbers}, end it with \end{linenumbers}. Or switch it on
%% for the whole article with \linenumbers after \end{frontmatter}.
\usepackage{lineno}

\usepackage{pgfgantt}
\usepackage{color, colortbl}

% create graphic elements 
\usepackage{tikz-qtree}
\usepackage{tikz}
\usetikzlibrary{matrix,shapes,arrows,positioning,quotes,shadows,decorations.pathreplacing,fadings}
\usepackage{graphicx}
\graphicspath{ {./images/} }

\usepackage{pgfplots}
\pgfplotsset{compat=1.17}


\title{Implementation of a Hard Random Instances Generator for the Open Shop Scheduling Problem}   % type title between braces
\author{Salah Eddine Bouterfif}         % type author(s) between braces

\begin{document}
\flushbottom
\maketitle
\thispagestyle{empty}

In their effort to calculate a new lower bound for the open shop problem, \cite{gp1999} showed that most open shop instances produced by random generators are easy and proposed a hard instances generator.\par

The random generator takes the number of jobs $n$ and the number of machines $m$ as input and requires three parameters $k$, $p$ and $f$.\\

We have adapted the generator from \cite{gp1999} to produce rectangle open shop instances. Algorithm~\ref{algo:hossp} describes how an instance is generated in two steps. The first step creates the $ n \times m $ matrix $P$ of processing times using the parameter $k$ such that all elements of $P$ are equal to $\dfrac{k}{m}$ (rounded down to the nearest integer). The second step performs a number of perturbations $p$ to the matrix $P$.\par

A parameter $f$ is used to calculate $v$: a fixed value \textit{to be subtracted} and $mv$: a \textit{maximum subtractable} value. 

The parameters $k$, $p$ and $f$ are described by:

\begin{itemize}
    \item $k$: integer number, such that the sum of each line of $P$ is equal to $k$. The value $k \mod m$ is added to the diagonal of $P$. 
    
    \item $p$: number of perturbations, such that for each perturbation, two task's processing times $P_{ij}$ and $ P_{kl}$, $i \neq k $ and $j \neq l$, are randomly selected from $P$, from which is subtracted the value of $v$.
    
    \item $f=\{f \in \mathbb{R} \mid 0<f<1\}$, such that:
    \subitem $mv =  \min\{P_{ij},P_{kl}\}-1$ (subtract 1 to avoid tasks of length 0).      
    \subitem $v = (f \cdot mv) + r $, with $r$: a random value between $0$ and $f \cdot mv$. 
\end{itemize}

The value subtracted from $P_{ij}$ and $P_{kl}$ is added to $P_{il}$ and $P_{kj}$ to keep the sum of all rows equal to $k$.

\begin{algorithm}[H]
    \caption{HOSSP: Hard Random Instance Generator.}\label{algo:hossp}
    \begin{algorithmic}[1]
      \Procedure{hossp}{$n,m, K, p, f$}
        \\  \Comment{Step 1: create a matrix $P$}
        \State $inc := 0$
        \If{$n=m$} \Comment{square problems}
            \ForEach {$i \leq n$ and $j \leq m$}
                    \State $P_{i,j} := \dfrac{K}{m}$
                \State $P_{i,i} := P_{i,i} + K \mod m$
            \EndForEach
        \Else{} \Comment{rectangle problems}
            \State $m^{\prime} = \min\{n, m\}$
            \ForEach{$i \leq n$ and $j \leq m$}        
                \If{$inc = m^{\prime}$}
                     $inc := 0$
                \EndIf
     
                \State $P_{i, j} := \dfrac{K}{m}$
                \If{$i \geq m^{\prime}$ and $j = inc$}
                    \State $P_{i, j} := P_{i,j} + K \mod m$
                \EndIf

                \If{$i < m^{\prime}$}
                    \State $P_{i, i} := P_{i,i} + K \mod m$
                \EndIf
                    \State $inc := inc+1$;
            \EndForEach        
        \EndIf
        \\ \Comment{Step 2: Add perturbations to the matrix $P$}
        \ForEach{$i \leq p$}   
        \\
        \While{$i \neq j$ and $k \neq l$}
            \State Randomly select $P_{i,j}, P_{k,l}$                                
        \EndWhile
        \\
        \State $mv =  \min\{P_{ij},P_{kl}\}-1$ \Comment{maximum subtractable value}
        \State $r:=random(0, mv)$ \Comment{ random value to subtract}
        \State $v = (f \cdot mv) + r $ \Comment{total value to subtract}
        \\
        \State$P_{i,j} := P_{i,j} - v$
        \State$P_{k,l} := P_{i,j} - v$
        \\
        \State$P_{i,l} := P_{i,l} + v$
        \State$P_{k,j} := P_{k,j} + v$
        \\  
        \EndForEach
        \\
      \EndProcedure
    \end{algorithmic}
  \end{algorithm}

For numerous generated instances the parameters $k$, $p$ and $f$ are randomly generated as follows:

\begin{itemize}
    \item $k$: random value between $ n \times m$ and $n \times m \times 100 $
    \item $p$: random value between $ n \times m$ and $ n \times m^{2} $
    \item $f$: random value between 0 and 1
\end{itemize}


In the case of $n \neq m$, $k \mod m$ is added the diagonal of top-down square sub-matrices of size $\min\{n,m\}$ of $P$. The main diagonal is used on the remaining rectangle ensuring the sum of all rows of $P$ are equal to $k$, this ensure that . 

\begin{thebibliography}{1}
    \bibitem{gp1999} Guéret, C., \& Prins, C. (1999). A new lower bound for the open-shop problem. Annals of Operations Research, 92, 165–183. https://doi.org/10.1023/A:1018930613891.

    %
\end{thebibliography}

\end{document}